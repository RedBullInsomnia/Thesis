\section{Problems encountered}
A master thesis is a major endeavour and these are rarely devoid of obstacles. During this year several elements obstructed the completion of this work : \begin{itemize}
\item V-Rep is a fine tool but the lack of a proper internal modelling tool really hindered this work as every major modification meant that the whole model had to be modified in Blender and re-imported into V-Rep. This created a lot of overhead work which contributed nothing of interest to this work. 

This drawback is generalized amongst all the simulators that we surveyed at the beginning of this report and we feel it should be addressed quickly by their creators. Nevertheless, we understand that a modelling tool such as Blender took years to create so we would not expect simulators to catch up any time soon.

\item We also learned of the importance of studying mechanisms carefully before using them. Though things might appear simple at first, subtle implementation details might change everything. A fine example of that is us melting the core of the motor inside a MX-28R servo during our tests because of our trust in the announced safety mechanisms. Needless to say, they proved insufficient and we should have examined the documentation more closely.
\end{itemize}


\section{Future work}
\subsection{Modelling}
As of now it is still uncertain if Blender shall continue to support the COLLADA format (as explained in \cite{blender_roadmap}). In the negative, another tool should be chosen to perform the modelling.

The springs also need some work, as of now they are just there as a proof of concept but their parameters will need to be tuned.

As of now, the model uses simplified inertias, in the belief that a controller should be able to correct minor differences in behaviour between the model and the actual robot. In the case, theses inertias need to be more accurate, we suggest to use Meshlab\footnote{\url{http://meshlab.sourceforge.net/}} to compute the inertias of those objects.

\section{Conclusion}
This is the conclusion to my work.
