\section{Conclusion}
In this work we applied rigid body simulation to the problem of designing a robust humanoid robot. Our work produced a model of a robot that respects the rules of the kidsize league of the RoboCup Soccer competition. Our simulations prove that it is able to stand up from a lying position, be it prone or supine. 

\section{Problems encountered}
A master thesis is a major endeavour and these are rarely devoid of obstacles. During this year several elements obstructed the completion of this work : \begin{itemize}
\item V-Rep is a fine tool but the lack of a proper internal modelling tool was a major thorn in the side as every major modification meant that the whole model had to be modified in Blender and re-imported into V-Rep. This created a lot of overhead work which contributed nothing of interest to this work. 

This drawback is generalized amongst all the simulators that we surveyed at the beginning of this report and we feel it should be addressed quickly by their creators. Nevertheless, we understand that a modelling tool such as Blender took years to create so we would not expect simulators to catch up any time soon.

\item We also learned of the importance of studying mechanisms carefully before using them. Though things might appear simple at first, subtle implementation details might change everything. A fine example of that is us melting the core of the motor inside a MX-28R servo during our tests because of our trust in the announced safety mechanisms. Needless to say, they proved insufficient and we should have examined the documentation more closely.

\item Choosing a physics engine was difficult because the field is quite fragmented. On one hand there exist well established commercial solutions, but they are focused on games and make some significant shortcuts whenever possible in order to be as fast as possible. On the other hand there exist a quantity of open-source physics engine but they are usually the work of one man and are poorly documented. It was hard to motivate the choice of Newton Dynamics on any other basis than 'it worked best'.
\end{itemize}

\section{Future work}
\subsection{Modelling}
While the model is in a usable state it could still be bettered and we suggest to begin with the items listed hereafter:
\begin{itemize}
\item \textbf{Springs.} Springs also need some work, as of now they are just there as a proof of concept but their parameters will need to be tuned.

\item \textbf{Inertias.} As of now, the model uses simplified inertias, in the belief that a controller should be able to correct minor differences in behaviour between the model and the actual robot. If these inertias need to be made more accurate, we suggest to use Meshlab\footnote{\url{http://meshlab.sourceforge.net/}} to compute the inertias of those objects.

\item \textbf{Model format.} It is still uncertain if Blender shall continue to support the COLLADA format, as explained mentioned in the development roadmap (\cite{blender_roadmap}). In the negative the choice should be made whether to continue using the COLLADA format and find another modelling software that supports it or to move on to another format (URDF for example, now that the robot model is defined).

\end{itemize}

\subsection{Routines}
Now that we have a simulator and a complete model of the robot, more routines can be created. 
\begin{itemize}
\item \textbf{Standing up from a supine position.} Even though the robot can roll from a supine to a prone lying position and use the standing from prone routine it would be faster to be able to stand from a supine position directly.

\item \textbf{Walking.} Being able to walk is the basic requirement for a robot to compete in RoboCup. A walking sequence is the last proof needed to be able to tell that the robot we designed is able to compete.

\item \textbf{Shooting a ball.} As soon as the robot is able to walk, the next step should be testing if it can shoot a soccer ball.
\end{itemize}

\subsection{Online simulation}
In parallel or after creating the routines aforementioned, the simulator should be used to test the high level control code of the robot. The interaction between the simulator and the control code will be the same but the control code will be much more complex than just a static sequence of orders.

