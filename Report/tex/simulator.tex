In this chapter we discuss the choice of V-rep as the simulation tool for this project. We begin by explaining the basics of rigid body dynamics simulation, take a survey of some of the existing simulators and finally test some of them.

\section{Simulation of rigid body dynamics}
TO DO

The list of physics simulating engines is quite long, but the most popular ones are, in no particular order :
\begin{enumerate}
\item Bullet
\item ODE
\item DART
\item Simbody
\item PhysX
\item Havok
\end{enumerate}

\begin{table}[htp]
\center
\begin{tabularx}{\textwidth}{@{} X X X X X X @{}}
\toprule
\textbf{Engine} & \textbf{License} & \textbf{Coordinates} & \textbf{Origin} & \textbf{Editor} &\textbf{Solver type}\\ 
\midrule
Bullet & Free & Maximal & Games & Blender & Iterative \\ 

ODE & Free & Maximal & Simplified robot dynamics, games & & Iterative\\ 

DART & Free & Generalized & Computer graphics, robot control & &\\

Simbody & Free & Generalized & Biomechamics & \\

PhysX & Proprietary & Maximal & Games & \\

Havok & Proprietary & Maximal & Games & \\
\bottomrule
\end{tabularx}
\caption{Features comparison\cite{engines_comparison}}
\label{table:specs}
\end{table}

\section{Available simulators}
An integrated simulation tool is preferred over a bare-bones physics engine because :
\begin{itemize}
\item time would be lost on creating 3D visualization
\item time would be lost on writing code to import model
\item time would be lost on debugging
\end{itemize}
and all that before the actual work could begin.

\textbf{Blender\cite{Bruyninckx04}
} : \begin{itemize}
\item Uses the Bullet engine
\item Scripting via Python, remote control possible through socket
\item Very complete modelling tool, in a class of its own.
\item Comment : Hard to use because of obscure simulation options and difficulties to correctly set inertias
\end{itemize}

\textbf{Gazebo} : \begin{itemize}
\item Can use Bullet, Simbody, DART or ODE.
\item Scripting via C++
\item Uses SDF format for models.
\item Comment : Hard to use because model must be in SDF format, which no CAD excepted 3dworks exports to. Furthermore, compiled language takes longer to test.
\end{itemize}

\textbf{V-Rep}: \begin{itemize}
\item Can use Bullet, Newton or ODE.
\item Internal scripting in LUA, provides remote API class.
\item Can import 3D collada models.
\item Comment : Best tool so far because model can be imported and the inertias are easy to control, simulation options as well.
\end{itemize}

\textbf{Matlab}: \begin{itemize}
\item Analytical modelling
\item Mathcode
\item No visualization
\item Comment : Not adapted because tedious modelling and no visualization and hard to handle friction and difficult to handle other objects.
\end{itemize}

\begin{table}[htp]
\center
\begin{tabularx}{\textwidth}{@{} l l X X X X @{}}
\toprule
\textbf{Simulator} & \textbf{License} & \textbf{Physics engine(s)} & \textbf{Integrated editor} & \textbf{Modelling}\\ 
\midrule
Blender & Free & Bullet & Fully fledged & Internal\\ 

V-REP & Free (educational license) & Bullet, ODE, Newton, Vortex(10s limit) & Limited & Can import .COLLADA\\

Gazebo & Free & Bullet, ODE, Simbody, DART & Limited & SDF format\\

Webots & Proprietary & ODE & None & SDF format\\

Matlab & Proprietary & None & None & Mathematical\\
\bottomrule
\end{tabularx}
\caption{Comparison of simulators}
\label{table:simulators_comp}
\end{table}

\section{Choice}
Out of Gazebo, V-Rep and Blender, V-Rep is chosen as the best tool because\begin{itemize}
\item Gives the choice between 3 engines, something blender cannot do
\item Makes it easier than Gazebo to create models, because Gazebo uses the URDF format
\item Gives better access than blender to the physical options of the simulation (intertias, timestep of engine)
\item It is multi-platform. 
\end{itemize}

The physics engine used is Newton Dynamics because simple tests showed it to be the most stable with a high number of joints, with the exception of Vortex but it requires a license to run more than $10s$.

While Blender is not used as the primary simulation tool, it is used in the early phases of the modelling because it is what it does best. More on it in the next chapter. 