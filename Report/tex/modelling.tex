This chapter covers the tools used in order to create a model of the robot, from the placement of the servos and joints to the incorporation of accelerometers.

\section{Blender}
The first stage of the modelling is done in Blender which is a lot more suited to this kind of work than V-Rep. 
Blender is used to do the following :
\begin{itemize}
\item place the servos, hinges and other elements in place.
\item simplify the servos, hinges into simple convex shapes with a low vertex count.
\item place position markers for the joints to be placed in V-Rep.
\end{itemize}

The model is finally exported in the COLLADA format.

\section{V-REP}
The model is finalized by :
\begin{itemize}
\item defining the mass and inertia of each piece(compiled in \cref{table:weights}) and enabling them for dynamic simulation.
\item adding joints between servos. For 2DOF joints, hinges are used as intermediates.
\item adding scripts to simulate sensors (COG, accelerometers).
\item adding springs on the legs through the use of prismatic and spheric joints.
\end{itemize}

\begin{table}[htp]
\center
\begin{tabularx}{\textwidth}{@{} X X l @{}}
\toprule
\textbf{Module} & \textbf{Weight [g]} & \textbf{Dimensions [mm x mm x mm]}\\ 
\midrule
Odroid C-2 & 40 & 85.0 x 56.0\\
Li-Po battery & 188 & 103.0 x 33.0 x 34.0\\
Mx-28R & 72 & 35.6 x 50.6 x 35.5\\
LI-USB30-M021C & 22 & 26.0 x 26.0 x 14.7\\
Frame Fr-07 & & \\
Frame Fr-101-H3 & 7 & \\
\bottomrule
\end{tabularx}
\caption{Weights and dimensions of the pieces of the robot}
\label{table:weights}
\end{table}

\subsection{Servos}
Servos are simulated by joints.

\subsection{Joints}
Spherical joint : 3DOF angular.

Prismatic joint : 1DOF linear.

Revolute joint : 1DOF angular.

\subsection{Sensors (accel, cog)}
The COG is computed through a script inside V-Rep, attached to a piece of the model and made available through the remote interface.

\subsection{Springs}