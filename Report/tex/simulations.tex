In this section we will explain how to use the simulator and how it was used to influence the design of the robot. Finally some simulations will be shown.

\section{Simulation setup}
The basic idea of the simulation is presented on \cref{fig:simulation_principles}.

\begin{figure}[htp]
\center
\includegraphics[width=0.6\textwidth]{figures/simulation_principles}
\caption[Simulation principles]{The idea is to let V-Rep simulate the robot and control with the actual code that runs on the robot.}
\label{fig:simulation_principles}
\end{figure}

The architecture of the system is explained on \cref{fig:remoteApi}. We choose to operate in the synchronous operating mode : before V-REP simulates a timestep it waits for a trigger, allowing us to precisely control the robot.

\begin{figure}[htp]
\center
\includegraphics[width=0.6\textwidth]{figures/remoteApiSynchronous}
\caption[Simulation interaction]{Typical interaction between the simulator and the control code. The simulation runs on two threads : the simulation and the server thread. The server threads can receive orders from a client thread which is controlled by a custom application of our own.}
\label{fig:remoteApi}
\end{figure}

\section{Applications}
\subsection{Static stability}
The first application is simply to build a model of the robot and test if it is able to stand upright on its own.

The modelling begins in Blender where pieces are simplified/made convex and placed to create the structure of the robot. The model is then exported (in COLLADA) and imported into V-Rep. 

In V-Rep the different elements of the robot are dynamically enabled and given mass, accordingly to the values listed in \cref{table:weights}. Then, joints (motor controlled with control loop activated) are added to simulate the behaviour of the servos. Their maximal torque is set to $1.6$, the maximum torque developed my Mx-28 servos as shown by our earlier experiments (\cref{table:exp1_results}). 

\begin{table}[htp]
\center
\begin{tabularx}{\textwidth}{@{} X X X l @{}}
\toprule
\textbf{Module} & \textbf{Weight [$g$]} &  \textbf{Density [$kg/m^3$]}& \textbf{Dimensions [$mm \times mm \times mm$]}\\ 
\midrule
Odroid C-2 & 40 &  & 85.0 x 56.0 x 10.0\\
Li-Po battery & 188 & 2304 & 103.0 x 33.0 x 24.0\\
Mx-28R & 72 & 1150 & 35.6 x 50.6 x 35.5\\
LI-USB30-M021C & 22 & 2200 & 26.0 x 26.0 x 14.7\\
Frame Fr-07 & & 1200 & \\
Frame Fr-101-H3 & 7 & 1200 & \\
\bottomrule
\end{tabularx}
\caption[Weights and dimensions of the pieces of the robot]{Weights and dimensions of the pieces of the robot. The density is useful for the automatic computation of the weight and inertia of the pieces in V-REP.}
\label{table:weights}
\end{table}

The springs on the leg are simulated two spherical joints and one prismatic joint set to spring-damper mode.

The servos of the robot are simply ordered to hold their initial angle and the simulation determines that the robot can indeed stand upright without any active stabilization.

\subsection{Standing up routines}
This section is heavily inspired by \cite{Stuckler06}

\subsection{Walking}

\section{Influence on robot's design}
The simulator helped shape the robot through simulations that unveiled serious design problems (inability to stand after a fall, inability to walk).

The first design is visible on \cref{fig:first_robot}. It was plagued by stability problems, overcomplicated arms and simulation difficulties. 
\begin{figure}[htp]
\center
\includegraphics[width=0.6\textwidth]{figures/robot1}
\caption[Initial robot design]{First robot design. Arms use 4 servos each, making it quite heavy.}
\label{fig:first_robot}
\end{figure}

The final design, visible on \cref{fig:final_robot} has better stability, wider movement possibilities and can stand up and walk more easily. 
\begin{figure}[htp]
\center
\includegraphics[width=0.6\textwidth]{figures/robot2}
\caption[Final robot design]{Final robot design. Arms now use 3 servos. The feet and the hips use a different configuration to have wider movement possibilities and bring down the center of gravity.}
\label{fig:final_robot}
\end{figure}

The final dimensions of the robot respect the rules of the contest:
\begin{itemize}
\item Height : $61.3cm$
\item Height of COM : $34cm$
\item Height of legs : $cm$
\item Height max is $< 1.5 \times 61.3$.
\item Foot area is $ cm^2$.
\end{itemize}