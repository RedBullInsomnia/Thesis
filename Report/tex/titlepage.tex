\begin{titlepage}


\begin{center}
\large
University of Liège - Faculty of engineering
\end{center}

\vfill

\begin{minipage}{0.5\textwidth}
\includegraphics[width=0.9\textwidth]{figures/ULg_logo_couleur.pdf}
\end{minipage}
\begin{minipage}{0.5\textwidth}
\huge
\textbf{Master thesis}\\\\
\normalsize
\textbf{Simulation of complex actuators}\\\\
Author : Hubert Woszczyk\\
Promotor : Pr. Bernard Boigelot\\
\end{minipage}

\vfill
\begin{center}
\large
Master thesis conducted for obtaining the\\ Master's degree in Electrical Engineering\\ by Hubert Woszczyk\\
\vspace*{8cm}
\normalsize
Academic year 2015-2016
\end{center}

\end{titlepage}

\newpage\null\thispagestyle{empty}\newpage

\thispagestyle{empty}
\begin{center}
    \Large
    \textbf{Simulation of complex actuators}
    
    \vspace{0.4cm}
    \large
    Hubert Woszczyk, under the supervision of Pr. Bernard Boigelot
    
    \normalsize
    Academic year 2015-2016\\
    Faculty of Applied Sciences\\
    Electrical Engineering
    
    \vspace{0.9cm}
    \textbf{Abstract}
\end{center}
The word \emph{robot} has been crafted by Czech writer Karel Čapek in his play R.U.R. (Rossum's Universal Robots) in the beginning of the XXth century and is derived from the slavic world \emph{robota} which means \emph{labour} or \emph{work}.


\clearpage
\pagenumbering{roman}
\setcounter{page}{1}
\chapter*{Acknowledgements}
My first thanks go to Prof. Bernard Boigelot who made it possible for numerous students, including me, to work in the passionate field of robotics. I also wish to thank him for his guidance, help and accessibility.

I am deeply grateful to my friends Elodie and Laurine for reading and correcting this manuscript. I also want to thank fellow students Grégory Di Carlo and Guillaume Lempereur with whom I had the pleasure of working together one last time.

Finally, I would like to express my sincere thanks to all those who helped me complete this master thesis.