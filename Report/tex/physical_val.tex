In this chapter we experiment with real servos in order to tune the parameters of the simulation.

\section{Problem statement}
Before using our simulator to test control algorithms it is useful to first verify that it gives physically accurate results. It would be a waste of time to conduct tests on a model that does not behave in the same way as the original does.

\section{Experimental set-up}
The set-up is explained in \cref{fig:exp_setup}. In later experiments a camera will be used to film the motion of the servos and compare it to the results of the simulation that is supposed to predict it.

\begin{figure}[htp]
\center
\includegraphics[width=0.6\textwidth]{figures/exp_setup}
\caption[Experimental setup]{Experimental setup : The MX-28 servos are powered by a DC generator and controlled by a laptop equipped with a USB2DYNAMIXEL(USB-2-DYN) device.}
\label{fig:exp_setup}
\end{figure}

\begin{figure}[htp]
\center
    \includegraphics[width = 0.8\textwidth]{figures/u2d}
    \caption[USB2DYNAMIXEL]{Picture of a USB2DYNAMIXEL device. It turns an USB port into a serial port (RS485, TTL or classic serial connector) that can be used to control Dynamixel manufactured servos. [Taken from \url{http://support.robotis.com/en/product/auxdevice/interface/usb2dxl_manual.htm}]}
    \label{fig:usb2dyn}
\end{figure}

\section{Experiment 1 \label{sec:exp1}}
The purpose of the first experiment is to test the torque : to that end, a frame is fixed onto a single servo and weighted. The setup is represented on \cref{fig:exp1}.

\begin{figure}[htp]
\center
    \includegraphics[width = 0.5\textwidth]{figures/exp1}
    \caption[Experimental setup for torque testing]{Experimental setup for torque testing. A weight $w$ of is suspended at a distance $d$ from the servo, resulting in a applied torque of $w \times g \times d$. The goal consists in finding the weight $w$ for which the servo is unable to lift the arm.}
    \label{fig:exp1}
\end{figure}

In our case, $d$ was equal to $22.5cm$ and we could reach a weight $w$ of $740g$ at $14.8V$. This equals to a torque of $1.64Ncm$. The complete results are listed in \cref{table:exp1_results}.
\begin{table}[htp]
\center
\begin{tabularx}{\textwidth}{@{}l X X X @{}}
\toprule
\textbf{Stall torque} & \textbf{@11.1V $[N.m]$} & \textbf{@12V $[N.m]$} & \textbf{@14.8V $[N.m]$}\\ 
\midrule
\textbf{Announced} & 2.1 & 2.5 & 3.1\\ 
\textbf{Experimental} & 1 & 1.2 & 1.6\\ 
\bottomrule
\end{tabularx}
\caption[Results of experiment 1]{Comparison of announced and experimental stall torques at different tested voltages. Announced values taken from \cite{mx_28_manual}}
\label{table:exp1_results}
\end{table}

\section{Experiment 2}
In this experiment we will test some simple dynamics. The setup is shown in \cref{fig:exp2}. The goal is to tune the P parameter of the PID controller inside V-Rep. In order to achieve this, we film the real servo going from $0\degree$ to $180\degree$ and measure the time it takes. 

Executing this manoeuvre at $12V$ with no speed limit imposed takes $620msec$.

We then reproduce the same manoeuvre inside V-Rep, measuring the time through code. We modify the proportional gain of the PID controller until we get the same time.

\begin{figure}[htp]
\center
    \includegraphics[width = 0.3\textwidth]{figures/exp2}
    \caption[Experimental setup dynamics testing]{Experimental setup for dynamics testing. One servo lifts the other one. The goal is the measure the time it takes to swing the arm from $0\degree$ to $180\degree$.}
    \label{fig:exp2}
\end{figure}