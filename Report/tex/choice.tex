In this chapter we discuss the choice of V-rep as the simulation tool for this project. We begin by explaining the basics of rigid body dynamics simulation, take a survey of some of the existing simulators and finally test some of them.

\section{Available simulators}
\subsection{Barebone physics engines}
The list of physics simulating engines is quite long, but the most popular ones are, in no particular order :
\begin{enumerate}
\item \textbf{Bullet :} The most popular open source physics engine as of now, used mainly for games(Rockstar Studios use it for their Grand Theft Auto games series) but also by a lot of tools (NASA tensegrity robotics toolkit, Blender, V-Rep). It supports rigid body and soft body simulation with collision detection and supports both rigid body and soft body constraints. Developed by Ewin Coumans, one of the creators of Havok.
\item \textbf{ODE :} Another popular open souce physics engine which is a little older than Bullet but still used by many roboticists for their simulations (V-Rep, Webots, Gazebo). This engine has been used by many commercial games over the years such Call of Juarez series or S.T.A.L.K.E.R.
\item \textbf{Newton :} Another open source engine, not quite as popular as Bullet and ODE but nevertheless used for commercial games (Amnesia, Soma). What makes it standout in the crowd is its deterministic constraint solver, as opposed to iterative solvers used by Bullet and ODE.
\item \textbf{PhysX :} A proprietary engine used primarily for games because its main focus is performance and not necessarily physical accurateness. Currently owned by Nvidia.
\item \textbf{Havok :} Another proprietary engine and a concurrent of PhysX, with virtually no difference between the two. Owned by Microsoft.
\end{enumerate}

\begin{table}[htp]
\center
\begin{tabularx}{\textwidth}{@{} l l l l l l @{}}
\toprule
\textbf{Engine} & \textbf{License} & \textbf{Coordinates} & \textbf{Origin} &\textbf{Solver type}\\ 
\midrule
Bullet & Free & Maximal & Games  & Iterative \\ 

ODE & Free & Maximal & Simplified robot dynamics, games & Iterative\\ 

PhysX & Proprietary & Maximal & Games & \\

Havok & Proprietary & Maximal & Games & \\
\bottomrule
\end{tabularx}
\caption{Features comparison\cite{engines_comparison}}
\label{table:specs}
\end{table}

\subsection{Simulators}
In this section we will speak of software that provide a higher level interface to the physics engines we presented earlier. That interface usually adds a visualization and some other useful features.
\begin{enumerate}

\item \textbf{Blender\cite{Bruyninckx04}} is 3D modelling software suite and as such has integrated the Bullet engine to help make more realistic animations. It features the ability to make Python scripts that use that engine to make games or physics simulations. It is cross platform.

\item \textbf{Gazebo} is the official simulator for the DARPA Atlas challenge. It features multiple physics engines (Bullet ,Simbody, Dart and ODE), allows custom plugins and uses the SDF format for its models. It is open source and runs natively on Linux systems but needs to be compiled in order to run on Windows or OSX.

\item \textbf{V-Rep :} is another simulator that lets you choose the physics engine(Bullet, ODE, Newton, Vortex) and it also allows custom plugins in the form of LUA scripts. It is cross-platform and uses its own format for storing models but can import standard formats(COLLADA, 3ds, etc...). It is free to use for educational purpose.

\item \textbf{Webots :} has virtually the same features as V-Rep but is not free.

\item \textbf{Matlab :} Not a dedicated robotics simulator per se but can be used to model the robot analytically and to write simulation code for it. 
\end{enumerate}

A summary of the features of each simulator is present on \cref{table:simulators_comp}.

\begin{table}[htp]
\center
\begin{tabularx}{\textwidth}{@{} l l X X X X @{}}
\toprule
\textbf{Simulator} & \textbf{License} & \textbf{Physics engine(s)} & \textbf{Integrated editor} & \textbf{Modelling}\\ 
\midrule
Blender & Free & Bullet & Fully fledged & Internal\\ 

V-REP & Free & Bullet, ODE, Newton, Vortex(10s limit) & Limited & Can import .COLLADA\\

Gazebo & Free & Bullet, ODE, Simbody, DART & Limited & SDF format\\

Webots & Proprietary & ODE & None & SDF format\\

Matlab & Proprietary & None & None & Mathematical\\
\bottomrule
\end{tabularx}
\caption{Comparison of simulators}
\label{table:simulators_comp}
\end{table}

\section{Tested software}
In order to choose the most suited tool some of the presented tools were installed and tested. Here are the results :
\begin{itemize}
\item Blender is pleasant to use because the robot's model can be easily changed inside it and the Python scripting allows fast development. Support for a socket allows an external program to control the robot. The internals of the physics are obscured and some interesting object properties, such as inertias, are hard to reach. It is also hard to change the simulation parameters making it difficult to obtain stable results when using a higher number of objects and constraints. Furthermore, support for the game engine, the basis of a simulation project is uncertain \cite{blender_roadmap}.

\item Gazebo is attractive because it has the support of DARPA and handles multiple physics engines. The main drawback lies in the modelling of the robot to be simulated. It does feature an internal modelling tool but it is too limited to be usable. The difficulty lies in the fact that it uses an xml file to store the parameters of the robot and the only tool that can export models to that format is 3ds max, a commercial product.

\item V-Rep also has multiple physics engines available and has a user-friendly interface. It also has an internal modelling tool but there is not much use for it since it allows the import of models in the COLLADA format. It also supports socket communication and even provides code for a client thread in the custom application. The options of the physics engines are also pretty accessible and lots of sensor types are natively supported by the simulator.
\end{itemize}

\section{Choice}
The first choice to be made is whether we go for a barebone physics engines or a simulator. The former has the advantage of being a highly customizable solution but a simulator provides much a physics engines does not :\begin{itemize}
\item 3D visualization
\item code handling models import
\item and many other
\end{itemize}
So a simulator is preferred. 

In the case of simulators we must eliminate Webots because it is not free for use. Bullet is nice for little game-like physics but its lack of access to the parameters of the simulation makes it really hard to use in practice. Gazebo suffers from the the use of the SDF format for its models. If a model exists, Gazebo is a good tool but when there is a no stably defined model it is handicap. So we are left with V-rep which is free, multi-platform, gives the choice of the physics engines as Gazebo does but also provides a better modelling workflow which allows us to modify the model of the robot. Another advantage is that this simulator is already used at the university. This choice is further confirmed by \cite{ivaldi2014tools} which shows that V-Rep is the highest noted tools amongst roboticists.

Inside V-Rep, we chose Newton Dynamics because simple tests showed it to be the most stable with a high number of joints, with the exception of Vortex but it requires a license to run more than $10s$.

Although Blender was not chosen as the primary simulation tool for the project, it shall be used as a modelling tool for the robot as its interface is in another class than V-Rep.